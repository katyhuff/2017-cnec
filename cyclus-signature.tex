%        File: 2017-cnec.tex
%     Created: Fri Jul 21 09:00 AM 2017 E
% Last Change: Fri Jul 21 09:00 AM 2017 E
%
\documentclass[letterpaper]{article}
\usepackage[top=1.0in,bottom=1.0in,left=1.0in,right=1.0in]{geometry}
\usepackage{verbatim}
\usepackage{amssymb}
\usepackage{graphicx}
\usepackage{longtable}
\usepackage{amsfonts}
\usepackage{amsmath}
\usepackage[usenames]{color}
\usepackage[hidelinks]{hyperref}
\def\thesection       {\arabic{section}}
\def\thesubsection     {\thesection.\alph{subsection}}
\usepackage{xspace}
\newcommand{\Cyclus}{\textsc{Cyclus}\xspace}%
\usepackage[acronym,toc]{glossaries}
\include{acros}
\makeglossaries

\author{Kathryn Huff\\
        Blue Waters Assistant Professor\\
        Department of Nuclear, Plasma, and Radiological Engineering\\
        University of Illinois at Urbana-Champaign\\
         \href{mailto:kdhuff@illinois.edu}{\texttt{kdhuff@illinois.edu}}}

\title{Fuel Cycle Simulation Capability for Signatures and Observables}
\begin{document}
\maketitle
\section{Project Description}

Nuclear power plants themselves are typically considered a lower proliferation risk than other fuel cycle facilities, particluarly those which produce or recycle fuel \cite{sciences_america_2009}. Accordingly, dynamic fuel cycle facility modeling is essential to assessment of proliferation risk, safeguardability, and material accountancy robustness. 


Students will develop fuel cycle facility models, or \emph{archetypes}, for use with the \Cyclus agent-based fuel cycle simulator \cite{huff_fundamental_2016}. These dynamically loadable, C++ shared-object libraries will capture material streams within myriad front-end and back-end fuel cycle facilities in support of nonproliferation-focused analyis.  
Specifically, one graduate student and up to four undergraduate students will improve existing enrichment, reactor, and reprocessing facility models to better capture material flows which impact material accountancy or qualify as observable signatures. 

Recent work \cite{mcgarry_mbmore_2017} has extended the \Cyclus fuel cycle
simulator\cite{huff_fundamental_2016} ecosystem to include archetypes dedicated
to modeling non-deterministic, nonproliferation-related treaty verification
behavior in a simple enrichment facility. The proposed effort will
significantly improve the fidelity of these capabilities for characterizing
\gls{SNM} signature and diversion detection
sensitivity\cite{mcgarry_mbmore_2017}.  \emph{Archetype} fidelity in this sense
refers to the ability to capture the dominant physics of the particular
facility technology at levels of detail appropriate to influence conclusions.

Student effort will emphasize conventional fuel cycle facilities. However,
students will also explore potential future technologies. For example, models
to inform monitoring strategies for MSR online fuel reprocessing concepts will
be approached. All of this work will support \Cyclus use cases focused on
`shadow' fuel cycles\cite{shadow}, detection sensitivity\cite{mgarry}, and
signature quantification\cite{sigs}.

The resulting simulation library will enable users of the \Cyclus simulator,
both scientists and students, to conduct more sophisticated dynamic simulations
relevant to non-proliferation. Specifically, increased mass flow fidelity at
the sub-facility level will enable users to answer novel questions concerning
intrinsic features of fuel cycle facility mass flows.

\section{Relevance to CNEC}
This work will improve existing capabilities for characterization of facilities
which process \gls{SNM}. Additionally, the research process will train undergraduate students in modeling and simulation of the nuclear fuel cycle as well as best practices for scientific computing \cite{wilson_best_2014}. 


Application of the proposed capabilities to signature detection and diversion sensitivity is relevant to the \gls{CNEC} \gls{SandO} thrust area goals.  Simultaneously, this work has a core focus on modeling and simulation that is quite relevant to the \gls{CNEC} \gls{SAM} thrust area.  

Finally, through direct collaboration with \Cyclus ecosystem community members at the national laboratories (including \gls{ORNL} and \gls{PNNL} in particular, students will be enabled to pursue internships with the national laboratories.

\section{Tasks and Deliverables}

A uranium enrichment facility based on the <+NAMEHERE+> multi-component formulation.

An \gls{LWR} model with emphasis on disambiguation of secondary effluents such as $^{85}Kr$.?

A reprocessing model with emphasis on disambiguation of non-raffinate effluent (such as?)?

These \emph{archetypes} will be used to demonstrate improved fidelity of 
\Cyclus analysis related to detection of signatures indicative of \gls{SNM} or 
\gls{SNM} diversion. 

The first task will be to implement a new multi-component enrichment archetype, 
enabling dynamic simulations which distinguish fuel cycles. This work will build upon a well-tested multi-component enrichment  

The second task will be to implement
The success of this project will be measured in journal articles and conference 
presentations, and student outcomes. Emphasis on practical preparation for 
technical modeling and simulation will benefit from the experience of the 
\gls{PI} in the area of scientific computing best practices 
\cite{scopatz_effective_2015,wilson_best_2014,huff_lessons_2017,huff_case_2017}.

All \emph{archetypes} developed in this work will be accompanied by sufficient documentation, testing, and integration with the simulator to enable reuse by other fuel cycle analysts\footnote{In accordance with the \Cyclus ecosystem, the products of this work will be BSD-3 licensed.}.


\cite{case_proliferation_2011}
Proliferation resistance and risk
Laser enrichment
quantification of uncertainty of proliferation risk assessment in nuclear fuel cycles
host-state proliferation, non-peaceful uses
for disambiguation of enrichment level, a model with detail at the cascade level is necessary



\section{Budget}
This work will require the effort of one graduate student for two years. If a 
suitable graduate student cannot be identified, it will require the effort of 
four undergraduate students, each working 20 hours per week, over the course of 
the 2017-2018 academic year. It will additionally require one week of summer 
salary for the PI. Finally, it will require travel funding for one 
student and the PI to present their work in a conference setting annually. 

As requested, the budget for this project is attached in Excel format.

\bibliographystyle{ieeetr}
\bibliography{2017-cnec-signatures}
\end{document}


