%        File: 2017-cnec.tex
%     Created: Fri Jul 21 09:00 AM 2017 E
% Last Change: Fri Jul 21 09:00 AM 2017 E
%
\documentclass[letterpaper]{article}
\usepackage[top=1.0in,bottom=1.0in,left=1.0in,right=1.0in]{geometry}
\usepackage{verbatim}
\usepackage{amssymb}
\usepackage{graphicx}
\usepackage{longtable}
\usepackage{amsfonts}
\usepackage{amsmath}
\usepackage[usenames]{color}
\usepackage[hidelinks]{hyperref}
\def\thesection       {\arabic{section}}
\def\thesubsection     {\thesection.\alph{subsection}}
\usepackage{xspace}
\newcommand{\Cyclus}{\textsc{Cyclus}\xspace}%
\usepackage[acronym,toc]{glossaries}
\include{acros}
\makeglossaries

\author{Kathryn Huff\\
        Blue Waters Assistant Professor\\
        Department of Nuclear, Plasma, and Radiological Engineering\\
        University of Illinois at Urbana-Champaign\\
         \href{mailto:kdhuff@illinois.edu}{\texttt{kdhuff@illinois.edu}}}

\title{Fuel Cycle Simulation Capability for Signature Modeling}
\begin{document}
\maketitle
\section{Project Description}

Fuel cycle facility models will be developed for use with the agent-based 
\Cyclus fuel cycle simulator \cite{huff_fundamental_2016}. These \Cyclus
\emph{archetypes} will significantly improve the fidelity of nascent \Cyclus
capabilities for characterizing \gls{SNM} signature and diversion detection
sensitivity\cite{mcgarry_mbmore_2017}.  Students will develop
\Cyclus-compatible, dynamically loadable, C++ shared-object libraries which
model:
\begin{itemize}
\item A uranium enrichment facility based on the <+NAMEHERE+> multi-component 
        formulation.
\item An \gls{LWR} model with emphasis on disambiguation of secondary effluents 
        such as $^{85}Kr$.?
\item A reprocessing model with emphasis on disambiguation of non-raffinate 
        effluent (such as?)?
\end{itemize}

These \emph{archetypes} will be used to demonstrate improved fidelity of 
\Cyclus analysis related to detection of signatures indicative of \gls{SNM} or 
\gls{SNM} diversion. 


The primary goal is to enable users of the \Cyclus simulator, both scientists
and students, to conduct more sophisticated dynamic simulations relevant to
non-proliferation. Specifically, increased mass flow fidelity at the facility
model (\emph{archetype}) level will enable users to answer novel questions
concerning intrinsic features of fuel cycle facility mass flows.
\emph{Archetype} fidelity in this sense refers to the ability to capture the
dominant physics of the particular facility technology at levels of detail
appropriate to influence conclusions.

\section{Relevance to CNEC}
This work will improve the \Cyclus capability to characterize facilities which process
\gls{SNM} and will train undergraduate students in modeling and simulation of
the nuclear fuel cycle. A focus on modeling and simulation is in line with the 
\gls{CNEC} \gls{SAM} thrust area. Simultaneously, application to signature and
diversion detection sensitivity is relevant to \gls{SandO} thrust area goals. 

\section{Tasks and Deliverables}

The first task will be to implement a new multi-component enrichment archetype, 
enabling dynamic simulations which distinguish fuel cycles. This work will build upon a well-tested multi-component enrichment  

The second task will be to implement
The success of this project will be measured in journal articles and conference 
presentations, and student outcomes. Emphasis on practical preparation for 
technical modeling and simulation will benefit from the experience of the 
\gls{PI} in the area of scientific computing best practices 
\cite{scopatz_effective_2015,wilson_best_2014,huff_lessons_2017,huff_case_2017}.

All \emph{archetypes} developed in this work will be accompanied by sufficient documentation, testing, and integration with the simulator to enable reuse by other fuel cycle analysts\footnote{In accordance with the \Cyclus ecosystem, the products of this work will be BSD-3 licensed.}.
\section{Budget}
This work will require the effort of one graduate student for two years. If a 
suitable graduate student cannot be identified, it will require the effort of 
four undergraduate students, each working 20 hours per week, over the course of 
the 2017-2018 academic year. It will additionally require one week of summer 
salary for the PI. Finally, it will require travel funding for one 
student and the PI to present their work in a conference setting annually. 

As requested, the budget for this project is attached in Excel format.

\bibliographystyle{ieeetr}
\bibliography{2017-cnec-signatures}
\end{document}


