%        File: 2017-cnec.tex
%     Created: Fri Jul 21 09:00 AM 2017 E
% Last Change: Fri Jul 21 09:00 AM 2017 E
%
\documentclass[letterpaper]{article}
\usepackage[top=1.0in,bottom=1.0in,left=1.0in,right=1.0in]{geometry}
\usepackage{verbatim}
\usepackage{amssymb}
\usepackage{graphicx}
\usepackage{longtable}
\usepackage{amsfonts}
\usepackage{amsmath}
\usepackage[usenames]{color}
\usepackage[
        naturalnames = true, 
        colorlinks = true, 
        linkcolor = Black,
        anchorcolor = Black,
        citecolor = Black,
        menucolor = Black,
        urlcolor = Blue
]{hyperref}
\def\thesection       {\arabic{section}}
\def\thesubsection     {\thesection.\alph{subsection}}

\author{Kathryn Huff\\
        Blue Waters Assistant Professor\\
        Department of Nuclear, Plasma, and Radiological Engineering\\
        University of Illinois at Urbana-Champaign\\
         \href{mailto:kdhuff@illinois.edu}{\texttt{kdhuff@illinois.edu}}}

\title{Simulation Capability for Alternative Plutonium Disposition}
\begin{document}
\maketitle

\section{Project Description}
4. Two to three paragraph description of the project

Cyclus archetypes modeling 

\section{Relevance to CNEC}
This work will improve our capability to characterize facilities which process
\gls{SNM} and will train undergraduate students in modeling and simulation of
the nuclear fuel cycle. 


Through an intimate mix of innovative research and development (R&D) and education activities, CNEC will enhance national capabilities in the detection and characterization of special nuclear material (SNM) and facilities processing SNM to enable the U.S. to meet its international nonproliferation goals, as well as to investigate the replacement of radiological sources so that they could not be misappropriated and used in dirty bombs or other deleterious uses.

\section{Tasks and Deliverables}
6. List of proposed Tasks and Deliverables completed per year

\pagebreak
\section{Budget}
7. Separate budget page with justification

\bibliographystyle{ieeetr}
\bibliography{paper}
\end{document}


